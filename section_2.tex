% !TEX root = ../thesis.tex

\section{Structure of scientific work}
In order to ensure a systematic structure, scientific work should be structured similarly to the structure described below. In addition to the cover page and the declaration of authorship, scientific papers usually include at least one index of contents, illustrations and tables as well as a list of abbreviations, followed by the actual main part and a bibliography. In general, it should be noted that a reasonable structure depends to a large extent on your topic and your chosen method. Thus, the structure of literature reviews, conceptual work and development work usually differs from each other.

\subsection*{Summary or Abstract}
The summary or abstract is detached from the actual elaboration and contains, in addition to a description of the task or objective, in particular the procedure, the methods applied as well as the essential results and summarises them as briefly as possible. The Summary or Abstract is not part of the structure of the work and therefore is not numbered.

\subsection*{Table of Contents}
The table of contents includes all first, second and third level headings listed later in the work. A deeper structure than in the third level is to be avoided in the context of scientific work. In this context, the bibliography and the appendix are included as first level headings in the structure and thus in the table of contents, but are not numbered. The table of contents is not included in the structure of the work and is explicitely not numbered!

\subsection*{Optional: I List of Figures}
The list of figures lists all the illustrations and figures used in the work, including the caption and the corresponding page numbers (preset in this template). The list of figures is the first list with an independent, otherwise Roman section numbering. This list only needs to be appended when necessary. In individual cases, it may be helpful to consult the work supervisor.

\subsection*{Optional: II List of Tables}
The list of tables lists all the tables used in the work, including their labels and the corresponding page numbers (preset in this template). This list only needs to be appended when necessary. In individual cases, it may be helpful to consult the work supervisor.

\subsection*{Optional: III List of Formulas}
The list of formulas lists all formulas used in the work, including the label and the corresponding page number. It is only necessary if there are at least 3 or more formulas used in the work. This list only needs to be appended when necessary. In individual cases, it may be helpful to consult the work supervisor.

\subsection*{Optional: IV List of Abbreviations}
The list of abbreviation lists all abbreviations used in the work, including their definition. This list only needs to be appended when necessary. In individual cases, it may be helpful to consult the work supervisor.

\subsection*{Optional: V List of Symbols}
The list of symbols lists all mathematical symbols used in the work, including their definition. This list only needs to be appended when necessary. In individual cases, it may be helpful to consult the work supervisor.

\subsection*{1 Introduction}
Within the introduction, the objective is formulated, placed in an superordinate context and distinguished from other topics. The most important terms of the topic must be placed within the context to be considered in the introduction while an thorough formulation is particularly important. In addition, information on the methodology can be provided. The structure of the work should be presented. 

Thus the first section contains the following aspects:

\begin{itemize}
	\item Description of the problem to be solved with this work
	\item Objective of the work and the delivered contribution to science
	\item Structure of the work
\end{itemize}

In the case of extensive scientific work, such as bachelor's and master's theses, the introduction should consist of one to three pages. In addition, it is useful to insert a figure at the end of the section which shows the structure, the argumentative sequence or important core statements of the work. Experience shows that the introduction should only be formulated in detail at the end of the work in order to avoid repeated changes to the main text.

\subsection*{2 Related Work}
Based on the introductory explanations, this section focuses on the contextualisation of the problem or objective described and the synthesis of the necessary fundamental principles. Necessary fundamental principles contain knowledge on topics that cannot be assumed to be engineering knowledge in the field of construction sciences or computer science.

In this context it is important to explain elementary terms, definitions and/or theories on the one hand and to present the state of research in the respective subject area on the other hand.
Accordingly, a summary is given of the scientific basis on which the scientific work to be written is based.

\subsection*{3 Methodology}
This section focuses on the methodology chosen to achieve the research objectives or to answer the research questions. It is made clear how the objective of the work formulated in the introduction is to be achieved or implemented using scientific methods.

\subsection*{4 Additional Main Section(s)}
In general, this main section or sections represents your results. The concrete structure of the scientific work depends largely on the task to be solved or the research questions to be answered, so that no detailed recommendations for this section can be given here, but exemplary sections could be requirements analysis, conception, development, evaluation for a development work or test setup, test execution, evaluation for an experimental work. For the specific structuring of the main section(s) a consultation with the supervisor of the work is helpful in any case.

\subsection*{5 Discussion}
The discussion serves to interpret or critically discuss the results achieved on the basis of the problem or question formulated in the introduction as well as in the context of current and relevant literature on the topic. In this context, unexpected results or contradictions to assumptions made or to the existing scientific literature will be specifically worked out.

\subsection*{6 Conclusion}
The conclusion completes the scientific work as a final statement. The objective, the methodology applied and the conclusions drawn are summarised once again and assessed in terms of the extent to which the introductory objective could be achieved. It is advisable to consider possible limitations of your work and to reflect critically on your approach. In addition, you should formulate in an outlook how further scientific work could tie in with its results and which priorities you see for further research work in the topic area of your work.

\subsection*{Bibliography}
In the bibliography, all sources used in the context of scientific work are specified in detail.

\subsection*{Appendix}
In principle, only those contents are included in the appendix which are not essential for the understanding of the content, but which may have an important relation to the own elaborations or an additional information character. These include, for example, questionnaires specially developed for the purpose of a study.

