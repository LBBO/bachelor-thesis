\section{Implementierung}
\subsection{Abhängigkeiten und Exklusitiväten}
Im Rahmen der Abhängigkeiten und Exklusivitäten von Erweiterungen zueinander gibt es mehrere Probleme, die gelöst werden müssen. Zum einen kann es schnell passieren, eine unmögliche Kombination von Bedingungen zu stellen. Außerdem muss nach der Auswahl einer Kombination von Erweiterungen geprüft werden, ob diese Kombination zulässig ist.

\subsubsection{Überprüfung der Umsetzbarkeit}


Ein triviales Beispiel hierfür wäre ein Paar von Abhängigkeiten $A$ und $B$, wobei $A$ und $B$ exklusiv zueinander sind und gleichzeitig $A$ von $B$ abhängt. Es können in diesem Beispiel niemals beide Anforderungen gleichzeitig erfüllt werden.

Es lassen sich aber auch kompliziertere Beispiele erzeugen. In Fig A wird eine Konstellation dargestellt, bei der $A$ von $B$ und $B$ von $C$ abhängt, aber $A$ zu $C$ exklusiv ist. Diese Kombination von Anforderungen lässt sich ebensowenig wie das erste Beispiel gleichzeitig erfüllen, ist aber komplizierter zu erfüllen. Die Ursache des Problems ist, dass Abhängigkeiten transitiv sind. Sind also die Erweiterungen $A$ von $B$ und $B$ von $C$ abhängig, so ist auch indirekt $A$ von $C$ abhängig.

Die bisherigen Beispiele zeigen Probleme auf, bei denen eine Erweiterung zugleich (indirekt) abhängig und exklusiv zu einer anderen Erweiterung ist. Es gibt jedoch auch eine weitere Kategorie von Problemen: sind zwei (indirekte) Abhängigkeiten einer Erweiterung zueinander exklusiv, so sind ebenfalls nicht alle Anforderungen gleichzeitig umsetzbar.

Aus diesen fachlichen Überlegungen heraus ergibt sich eine erste Lösung. Für jede Erweiterung $E$ sind zunächst alle direkten sowie indirekten Abhängigkeiten zu bestimmen. Für alle diese Abhängigkeiten ist dann zu prüfen, dass zum einen die Abhängigkeit $A$ nicht zu $E$ exklusiv ist, aber auch dass $A$ zu keiner weiteren (indirekten) Abhängigkeit $A'$ von $E$ exklusiv ist.

Bei näherer Betrachtung dieser Lösung lässt sich jedoch einiges an Ineffizienz feststellen. Zum einen werden für jede Erweiterung erneut die transitiven Abhängigkeiten berechnet. Aufgrund eben dieser Transitivität sind jedoch die Berechnungen für alle Erweitungen redundant, die Abhängigkeit einer anderen Erweiterung sind. Im schlimmsten Fall (nämlich wenn eine von $n$ Erweiterung alle anderen als (indirekte) Abhängigkeiten hat) ist also eine Berechnung nötig und $n - 1$ Berechnungen werden unnötigerweise durchgefüht. Aus ähnlichem Grund ist auch die Überprüfung der Exklusitivät zweier (indirekter) Abhängigkeiten häufig überflüssig.

Die Lösung dieser Probleme ergibt sich aus einer theoretischeren Betrachtung des Problems. Wie bereits aus der Verwendung des Begriffs der Transitivität hervorgeht, lassen sich Abhängigkeit und Exklusitivät als mathematische Relationen über der Menge aller Erweiterungen auffassen. Hierbei ist besonders hervorzuheben, dass die Exklusivität symmetrisch ist (ist $A$ zu $B$ exklusiv, so ist auch $B$ zu $A$ exklusiv) während Abhängigkeit nicht symmetrisch ist (im Gegenteil: bei der anfänglichen Analyse von möglichen Bibliotheken, Frameworks etc. ergab sich, dass Abhängigkeit nie symmetrisch zu sein scheint). Allerdings ist Exklusivität a priori nicht transitiv (auch, wenn $A$ zu $B$ und $B$ zu $C$ exklusiv ist, können $A$ und $B$ zusammen verwendet werden), während Abhängigkeit sehr wohl transitiv ist (wie bereits erläutert).

Vor diesem Hintergrund lässt sich erkennen, dass die Bestimmung der (indirekten) Abhängigkeiten der Bestimmung der Transitiven Hülle gleichkommt. Diese kann mittels des Floyd-Warshall-Algorithmus (in der Warshall-Variante) berechnet werden \cite{warshal1_algorithm}. Hierfür muss zunächst ein gerichteter Graph erzeugt werden, in den alle bekannten Abhängigkeiten als Kante eingefügt werden. Von diesem Graphen wird dann die transitive Hülle bestimmt, in der dann zwei Knoten $A$ und $B$ genau dann durch eine (von $A$ nach $B$ gerichtete) Kante verbunden sind, wenn die zugehörige Erweiterung $A$ von $B$ abhängt.

Auch die Relation der Exklusivität lässt sich in einen Graphen überführen. Auch in diesem Graphen gibt es pro Erweiterung einen Knoten und jede Exklusivität wird als Kante dargestellt. Aufgrund der Symmetrie der Exklusivität kann dieser Graph aber ungerichtet sein. Das Problem der Überprüfung der Restriktionen reduziert sich nun darauf, sicherzustellen, dass es für jede zwischen zwei Knoten in maximal einem der beiden Graphen eine Kante gibt, wobei die Richtung keine Rolle spielt (denn wenn die Knoten exklusiv zueinander sind, darf keiner von dem anderen abhängig sein).

Anders formuliert, darf es im Graphen der Exklusivitäten keine Kante $\{A, B\}$ geben, für die in der transitiven Hülle der Abhängigkeiten die Kante $(A, B)$ oder die Kante $(B, A)$ existiert. Aufgrund dessen, dass die transitive Hülle gerichtet ist, gibt es darin doppelt so viele Kanten, wie in dem Graphen der Exklusivitäten, weshalb diese Richtung einfacher zu prüfen ist, als die Umkehrung.

Der Algorithmus von Floyd-Warshall sorgt dafür, dass diese Überprüfung eine asymptotische Laufzeit von $\mathcal{O}(n^3)$ hat. Da die oben beschriebenen Probleme in beliebiger Tiefe von Abhängigkeiten auftreten können, ist jedoch die Bestimmung der transitiven Hülle nicht vermeidbar. Allerdings ist damit zu rechnen, dass die Anzahl aller Erweiterungen stets kleiner als $100$ sein wird (andernfalls würde die Benutzbarkeit des Programms möglicherweise stark eingeschränkt). Daher ist diese Laufzeit in diesem Fall als unbedenklich einzustufen.