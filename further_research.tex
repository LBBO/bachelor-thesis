\section{Ausblick}
\label{further_research}

Wie bereits an einigen Stellen erwähnt wurde, gibt es viele Aspekte, auf die sich weitere Forschung konzentrieren könnte. In keiner besonderen Reihenfolge werden hier einige dieser Aspekte aufgelistet.

Da im Rahmen dieser Arbeit die Entwicklung von Erweiterungen lediglich exemplarisch erfolgt ist, wäre es sinnvoll, weitere Erweiterungen zu ergänzen. Insbesondere fehlt Unterstützung für Vue.js. Da das Vue-\gls{CLI} jedoch große Ähnlichkeiten zum Angular-\gls{CLI} aufweist, müssen für die Vue-Erwieterung keine neuen Konzepte erarbeitet werden.

Des weiteren fehlen Erweiterungen für weitere Werkzeuge und Bibliotheken. Größtenteils sind auch hier keine neuen Konzepte erforderlich, sodass sich an bereits implementierten Erweiterungen orientiert werden kann. Ausnahme hiervon sind jedoch, wie bereits im vorigen Kapitel beschrieben, die Erweiterungen für Werkzeuge und Bibliotheken zum Schreiben von automatisierten Tests.

Hier könnte beispielsweise eine Abstraktion über gängige Werkzeuge und Bibliotheken entwickelt werden, sodass andere Erweiterungen an einer Stelle Tests definieren können, die mittels der Abstraktion nur auf gemeinsame Features aufbauen. Aus diesen abstrahierten Definitionen könnten dann für jede Testbibliothek entsprechende Dateien erzeugt werden.

Auch im Bereich der automatisierten Tests für \gls{GWA} ist weitere Arbeit zu leisten. Die Aussagekraft der Ende-zu-Ende-Tests ist bisher nur gering, da lediglich die Baubarkeit (d.h. der Erfolg des \verb/npm run build/-Befehls) bestimmter Konfigurationen überprüft wird. Durch eine Einbindung inhaltlicher Tests für alle verwendeten Erweiterungen könnte diese Aussagekraft vergrößert werden. Außerdem könnte untersucht werden, ob die Tests parallelisiert ablaufen können, um die aktuell bei ca. 20 Minuten liegende Laufzeit zu verkürzen. Da ein großer Teil dieser Laufzeit von der Installation der \gls{npm}-Pakete kommt, könnte ein weiterer Ansatz zur Laufzeitverbesserung auch das Zwischenspeichern dieser Pakete sein. Eine weitere Verbesserungsmöglichkeit bietet die Erzeugung der Liste der zu überprüfenden Konfigurationen. Aktuell muss diese manuell gepflegt werden; es wäre jedoch sinnvoll, diese basierend auf der Menge aller verfügbaren Erweiterungen automatisch zu erzeugen. So müsste diese Liste nicht bei jedem Hinzufügen einer neuen Erweiterung oder bei jeder Modifikation der stellbaren Fragen modifiziert werden.

Im Rahmen der Erweiterungen besteht auch an allgemeinen Stellen Verbesserungsbedarf. Zum einen wäre es hilfreich, wenn Nutzenden nicht erst beim Versuch der Absendung einer Antwort auf eine Frage Feedback zur Validität der Antwort gegeben würde, sondern dies schon beim Ausfüllen der Antwort geschehen könnte. Zum anderen ist die manuelle Festlegung der Installationsreihenfolge der Erweiterungen suboptimal, da bei ihrer Erzeugung schwer zu findende Fehler unterlaufen können, die nur bei sehr bestimmten Kombinationen von Erweiterungen auftreten. Sinnvoller wäre hier eine automatische Generation basierend auf Restriktionen, die pro Erweiterung festgelegt werden können. Wie bereits erwähnt, müsste hierfür vorsichtig ein System von Restriktionen ausgearbeitet werden.

Die momentan sehr auf die Installation bestimmter Bibliotheken oder Werkzeuge konzentrierten Fragen könnten zukünftig um weitere Konfigurationsfragen erweitert werden. So wäre es möglich, beispielsweise ESLint von Anfang entsprechend Präferenzen des / der Nutzenden zu konfigurieren, sodass diese Konfiguration nicht noch nach der Installation vorgenommen werden muss.

Insbesondere, wenn derartig umfangreiche Konfigurationen getroffen werden können, aber auch schon im Rahmen der reinen Installation von Projekten wäre es für die wiederholte Nutzung von \gls{GWA} hilfreich, wenn nach dem Beantworten aller Fragen die Antworten als \glqq Preset\grqq\ gespeichert werden könnten, sodass bei späteren Aufrufen zuvor erstellte Presets verwendet werden können. Dies würde die wiederholte Verwendung einer Lieblingskonfiguration erleichtern. Dabei sollte jedoch auch darauf geachtet werden, dass vor der Installation noch Anpassungen an das gewählte Preset vorgenommen werden können, sodass besondere Eigenheiten des zu erzeugenden Projektes berücksichtigt werden können.

Zukünftige Forschung könnte sich ebenso auf die weitere Verallgemeinerung von \gls{GWA} konzentrieren. Eine mögliche Verallgemeinerung wäre die Entfernung der Notwendigkeit eines Frameworks. So könnten beispielsweise frameworklose Webprojekte oder auch Node.js-Projekte erzeugt werden. Außerdem lässt sich evaluieren, ob eine Erweiterung von \gls{GWA} auf andere Programmiersprachen sinnvoll umsetzbar ist. Gerade im von konkreten Erweiterungen unabhängigen Teil des Codes müssten hierfür weitere Abstraktionen getroffen werden, aber erste Grundlagen für eine solche Entwicklung sind beispielsweise durch die Abstraktion des Paketmanagers bereits geschaffen worden.