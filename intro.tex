\section{Einleitung}
Entwickelnde haben zu Beginn eines neuen Projektes immer erst die Aufgabe, sich eine geeignete Architektur für das Projekt zu überlegen, die sie dann möglichst von Anfang an verfolgen können. In den meisten Programmiersprachen und Umfelden, insbesondere in der Webentwicklung, ist ein Teil dieser Aufgabe die Auswahl von Bibliotheken und Werkzeugen, die bei der Programmierung von Hilfe sein können.

Unter diesen Bibliotheken und Werkzeugen gibt es viele, die besonders häufig verwendet werden oder in bestimmten Kombinationen empfohlen werden. Beispielsweise verwenden 60,4\% der von \glqq State of JS 2020\grqq\ Befragten die Bibliothek Axios\footnote{\url{https://www.npmjs.com/package/axios}} (ein HTTP-Client, die für Browser und Node.js dasselbe Interface bietet)\cite{stateofjs}. Unter den Befragten nutzen 89,6\% das Werkzeug \gls{npm}\footnote{\gls{npm} genießt jedoch als offizieller Node.js-Paketmanager eine Sonderstellung. Daher ist diese Zahl nur begrenzt mit anderen Verwendungsquoten vergleichbar.}, 82,3\% nutzen ESLint\footnote{\url{https://www.npmjs.com/package/eslint}} (ein Werkzeug zur statischen Codeanalyse) und 70,9\% nutzen Prettier\footnote{\url{https://www.npmjs.com/package/prettier}} (ein Codeformatierer).

Die genaue Konstellation, in der diese Bibliotheken und Werkzeuge verwendet werden, unterschieden sich jedoch von Projekt zu Projekt. Sie hängen primär von den Anforderungen des Projekts, aber auch von den persönlichen Präferenzen der beteiligten Entwickelnden ab.

Nachdem entschieden wurde, welche Konstellation von Bibliotheken und Werkzeugen zu verwenden ist, müssen diese installiert werden und miteinander verknüpft werden, sodass sie alle zusammen funktionieren. Dieser Schritt unterscheidet sich von klassischen Entwicklungsaufgaben, da hierbei Probleme wie Laufzeitanalyse oder Graphentheorie keine Rolle spielen. Stattdessen müssen Konfigurationsdateien erzeugt werden, Objekte müssen in gewissen Reihenfolgen initialisiert werden und Werkzeuge müssen in den Kompilierprozess eingebunden werden, der je nach Konstellation der Werkzeuge zunächst eingerichtet werden muss.

Ziel dieser Arbeit ist es, diese initiale Projektgenerierung zu automatisieren. Über ein \gls{CLI} sollen zunächst einige Fragen gestellt werden, durch die festgestellt wird, welche genaue Konstellation von Bibliotheken und Werkzeugen installiert werden soll. Entsprechend dieser Spezifikation soll dann ein Projekt erzeugt werden, in dem keine weitere Verdrahtung mehr notwendig ist und in dem sofort alle Bibliotheken und Werkzeuge verwendet werden können.

Es gibt bereits mehrere Projekte, die sich ebenfalls dieser Aufgabe widmen und auf die im späteren Verlauf der Arbeit näher eingegangen wird. Diese haben jedoch den Nachteil, dass sie nur begrenzte Unterstützung für Bibliotheken und Werkzeuge liefern. Manche der Projekte bieten eine Möglichkeit zur Erweiterung um neue Bibliotheken und Werkzeuge an, jedoch sind derartige Erweiterungen nicht schon bei der Projektinitialisierung verwendbar und müssen daher in weiteren Schritten nach der Initialisierung ausgeführt werden.

Neben der beschriebenen Erleichterung bieten derartige \gls{CLI}s auch den Vorteil, dass sie Einsteigenden bei der Entdeckung neuer Werkzeuge und Bibliotheken helfen können. Im Jahr 2019 gab es bereits über 1,3 Millionen \gls{npm}-Pakete \cite{npm_package_count}, die einen Überblick insbesondere am Anfang erschweren.

Durch ein \gls{CLI} kann eine Vorauswahl dieser Pakete getroffen werden, die leichter von Einsteigenden überblickt werden kann. Zudem können der Vorauswahl weitere Informationen wie Abhängigkeiten zwischen Bibliotheken oder Links zu zugehörigen Dokumentationsseiten angehängt werden, die das initiale Erstellen eines Projektes zusätzlich erleichtern.

Im Rahmen dieser Arbeit wird daher ein solches \gls{CLI} namens \gls{GWA} entwickelt. Es soll in der Lage sein, leicht verständliche Fragen zu stellen und basierend auf den Antworten ein funktionsfähiges Projekt zu erzeugen. Bei der Konzeptionierung und Entwicklung wird ferner darauf geachtet, dass \gls{GWA} leicht erweitert werden kann, um die Einbeziehung zukünftiger oder zum Entwicklungszeitpunkt nicht berücksichtigter Bibliotheken und Werkzeuge zu ermöglichen.

Bevor mit der Konzeptionierung von \gls{GWA} begonnen wird, wird der Stand der Technik erläutert. Dies umfasst neben einigen Grundlagen, die zum besseren Verständnis der Arbeit beitragen, eine Vorstellung der bereits erwähnten Projekte mit ähnlicher Zielsetzung.

Daraufhin wird analysiert, welche Anforderungen genau von \gls{GWA} erfüllt werden sollen. Hierfür werden zunächst die Vor- und Nachteile der bereits existierenden Projektgeneratoren analysiert. Daraufhin wird eine Auswahl der zu verwendenden Technologien getroffen und es werden erste Pläne zur Gestaltung des Ablaufs und des Datenmodells angefertigt.

Der Fokus der Arbeit liegt auf der Implementierung von \gls{GWA}. In diesem Zusammenhang wird das allgemeine Vorgehen bei der Entwicklung erläutert. Es werden einige zentrale Probleme untersucht, für die verschiedene Lösungsoptionen entwickelt und diskutiert werden und von denen anschließend eine ausgewählt und umgesetzt wird.

Abschließend wird evaluiert, ob die gesetzten Ziele erfüllt werden konnten. Es wird diskutiert, welche Probleme weiterhin bestehen und womit sich zukünftige Forschung auseinandersetzen könnte.