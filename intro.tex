\section{Motivation}
Entwickelnde haben zu Beginn eines neuen Projektes immer erst die Aufgabe, sich eine geeignete Architektur für das Projekt zu überlegen, die sie dann möglichst von Anfang an verfolgen können. In den meisten Sprachen, insbesondere in der Webentwicklung ist ein Teil dieser Aufgabe die Auswahl von Bibliotheken und Werkzeugen, die bei der Programmierung von Hilfe sein können.

Unter diesen Bibliotheken und Werkzeugen gibt es viele, die besonders häufig verwendet werden oder in bestimmten Kombinationen empfohlen werden. Beispielsweise verwenden 60,4\% der von "State of JS 2020" Befragten die Library Axios\footnote{\url{https://www.npmjs.com/package/axios}} (ein HTTP-Client, die für Browser und Node.js dasselbe Interface bietet)\cite{stateofjs}. 89,6\% der Befragten nutzen das Tool \gls{npm} (wobei dies als offizieller Node.js-Paketmanager eine Sonderstellung genießt), 82,3\% nutzen ESLint\footnote{\url{https://www.npmjs.com/package/eslint}} (ein Werkzeug zur statischen Codeanalyse) und 70,9\% nutzen Prettier\footnote{\url{https://www.npmjs.com/package/prettier}} (ein Codeformatierer).

Geläufige Kombinationen von Werkzeugen oder Bibliotheken ergeben sich oft aus der Installationsanleitung bzw. aus der offiziellen Dokumentation. So 