\RequirePackage[dvipsnames,svgnames]{xcolor} % Change font colors
\documentclass[%
  ngerman,        % (ngerman/english) internationalisation
%  listoffigures,  % optional listings of figures, tables and listings
%  listoftables,
%  listoflistings,
%  twoside,        % double sided layout
]{iib_thesis}

\newcommand{\missingQuote}{\textcolor{red}{\underline{\textbf{\uppercase{zitat fehlt}}}}}

% Add `diff` language for lstlisting
  \definecolor{diffstart}{named}{Grey}
  \definecolor{diffincl}{named}{Green}
  \definecolor{diffrem}{named}{OrangeRed}

  \lstdefinelanguage{diff}{
    basicstyle=\ttfamily\small,
    morecomment=[f][\color{diffstart}]{@@},
    morecomment=[f][\color{diffincl}]{+\ },
    morecomment=[f][\color{diffrem}]{-\ },
    numbers=left,
  }

\title{Entwicklung eines erweiterbaren Kommandozeileninterfaces zur Generierung von Web-Applikationen}
\author{Michael David Kuckuk}
\degree{Bachelor}
\studyprogram{Angewandte Informatik}
\date{\today}
\matno{108 017 207 503}
\firstadviser{Prof. Dr.-Ing. Markus König}
\secondadviser{Stephan Embers, M.\,Sc.}
\myabstract{
The ABSTRACT is to be a fully-justified text following the title page. The text will be formatted in 12\,pt, single-spaced Computer Modern. The title is ``Abstract'', set in 12\,pt Computer Modern Sans Serif, centered, boldface type, and initially capitalized. Writing the abstract in English is mandatory even if the thesis itself is written in German. The length of the abstract should be roughly about 200 words but must not exceed 250 words.

As usual, the abstract should clearly summarize the aim, the background and the results of your thesis so that an interested reading can decide if he or she wants to further read your (interesting) written report. Also, the abstract should be kept simple. That is, do not include tables, figures or cross references in the abstract. You will have plenty of time in your thesis to explain things more clearly.
}

\begin{document}

  \newacronym{CLI}{CLI}{Command Line Interface}
\newacronym{CRA}{CRA}{create-react-app}
\newacronym{npm}{NPM}{Node Package Manager}
\newacronym{MVC}{MVC}{Model View Controller}
\newacronym{DOM}{DOM}{Document Object Model}
\newacronym{VCS}{VCS}{Versionskontrollsystem bzw. Engl.: Version Control System}
\newacronym{JSON}{JSON}{JavaScript Object Notation}
\newacronym{GWA}{GWA}{generate-web-app}
\newacronym{VM}{VM}{Virtuelle Maschine}
\newacronym{TDD}{TDD}{Test-driven Development}

  \section{Motivation}
Entwickelnde haben zu Beginn eines neuen Projektes immer erst die Aufgabe, sich eine geeignete Architektur für das Projekt zu überlegen, die sie dann möglichst von Anfang an verfolgen können. In den meisten Sprachen, insbesondere in der Webentwicklung ist ein Teil dieser Aufgabe die Auswahl von Bibliotheken und Werkzeugen, die bei der Programmierung von Hilfe sein können.

Unter diesen Bibliotheken und Werkzeugen gibt es viele, die besonders häufig verwendet werden oder in bestimmten Kombinationen empfohlen werden. Beispielsweise verwenden 60,4\% der von \glqq State of JS 2020\grqq\ Befragten die Library Axios\footnote{\url{https://www.npmjs.com/package/axios}} (ein HTTP-Client, die für Browser und Node.js dasselbe Interface bietet)\cite{stateofjs}. 89,6\% der Befragten nutzen das Tool \gls{npm} (wobei dies als offizieller Node.js-Paketmanager eine Sonderstellung genießt), 82,3\% nutzen ESLint\footnote{\url{https://www.npmjs.com/package/eslint}} (ein Werkzeug zur statischen Codeanalyse) und 70,9\% nutzen Prettier\footnote{\url{https://www.npmjs.com/package/prettier}} (ein Codeformatierer).

Geläufige Kombinationen von Werkzeugen oder Bibliotheken ergeben sich oft aus der Installationsanleitung bzw. aus der offiziellen Dokumentation. So 
%  % !TEX root = ../thesis.tex

\section{Layout}

The layout of the title page is as shown on the title page of this document. It contains the thesis' title bordered by horizontal edges, information about the student, and information about the supervisors.

All pages except the title page and the abstract are numbered. Numbers are placed on the pages' outer sides. Indicies are numbered in roman numerals whereas the rest is numbered in arabic numerals.

Sections are numbered automatically. You can write sections and subsections to structure your text (using the corresponding LaTeX keywords). If necessary, you can further use a ``subsubsection'' keyword, but that should be enough.

  \subsection{Structure}
  The structure of a thesis is as follows:

  \begin{itemize}
      \item Title page
      \item Declaration
      \item Abstract
      \item Table of Contents
      \item Thesis Specific Structure of Sections
      \item Lists of Figures, Tables, Listings (only if absolutely necessary)
      \item References
		\item Appendex (only if necessary)
  \end{itemize}

\section{Formatting}

	All text is in single-column format. It is fully-justified 12\,pt Computer
Modern. The Table of Contents--depth is set to 2, so even if there are deeper
subsections, they not included in the Table of Contents.

  \subsection{Figures and Tables}

  Figures can be placed in your text to show some interesting stuff. In the simplest case, use the floating ``figure'' environment with a good caption (see Fig. \ref{fig:example}).

  \begin{figure}
		\centering
		\includegraphics[width=0.5\textwidth]{cat.jpeg}
      \caption{A picture of a cat, the second most popular Internet theme.}
		\label{fig:example}
  \end{figure}

  Figures should not contain code, use listings (see section \ref{subsec:listings}) instead. Captions contain a full description of the figure including parameters and color codes if not given within the figure. All figures have to be referenced in the text. Subfigures are alphabetically enumerated where the letters are parenthesized.  Caption are written \textit{below} figures, but \textit{above} tables (see Table \ref{tab:example}).

  \begin{table}
	  \centering
	  \caption{A simple table example using the \textit{tabularx} package}
	  \begin{tabularx}{0.4\textwidth}{cr}
		  \textbf{Pet Type} & \textbf{Price [EUR]}\\
		  \hline
		  cat & 33.00 \\
		  dog & 22.00 \\
		  guinea pig & 11.00 \\
		  \hline
	  \end{tabularx}
	  \label{tab:example}
  \end{table}

  \subsection{Equations}
  All equations have to be referenced in the text and should be written as LaTeX math code and not included as an image. See any LaTeX math tutorial how to do this, for example the Wikipedia book on this subject: \url{https://en.wikibooks.org/wiki/LaTeX/Mathematics}.

  \subsection{Listings}
  \label{subsec:listings}

	All listings should be referenced in the text. There is a \textit{listings} package that can you can use to clearly show some code snippets in your text. See the CTAN website for more details: \url{https://ctan.org/pkg/listings}. In any case, listings should be used sparsely, use pseudo-code instead of source code where applicable.

  \subsection{References}

	When you refer to another publication or a website in your thesis, you must give a proper reference using Bibtex. You can cite a book, for example the LaTeX book by Leslie Lamport \cite{lamport2005}; or a conference paper, for example one given at Oulu, Finnland \cite{hilfert_koenig_2015}; or a paper in a journal, for example a paper in ``Visualization in Engineering'' \cite{hilfert_koenig_2016}.

	Citing a website (URL) is OK, but include as much information as possible, not just the URL \cite{wiki:xxx}.

	\subsection{Appendix}

	\begin{itemize}
		\item
			Lots of pictures and diagrams...
		\item
			Lots of important pieces of program code...
		\item
			Long tables of important, but boring, numbers...
		\item
			Anything else which is important, but will hinder the reading in
			the main text...
	\end{itemize}

%  % !TEX root = ../thesis.tex

\section{Structure of scientific work}
In order to ensure a systematic structure, scientific work should be structured similarly to the structure described below. In addition to the cover page and the declaration of authorship, scientific papers usually include at least one index of contents, illustrations and tables as well as a list of abbreviations, followed by the actual main part and a bibliography. In general, it should be noted that a reasonable structure depends to a large extent on your topic and your chosen method. Thus, the structure of literature reviews, conceptual work and development work usually differs from each other.

\subsection*{Summary or Abstract}
The summary or abstract is detached from the actual elaboration and contains, in addition to a description of the task or objective, in particular the procedure, the methods applied as well as the essential results and summarises them as briefly as possible. The Summary or Abstract is not part of the structure of the work and therefore is not numbered.

\subsection*{Table of Contents}
The table of contents includes all first, second and third level headings listed later in the work. A deeper structure than in the third level is to be avoided in the context of scientific work. In this context, the bibliography and the appendix are included as first level headings in the structure and thus in the table of contents, but are not numbered. The table of contents is not included in the structure of the work and is explicitely not numbered!

\subsection*{Optional: I List of Figures}
The list of figures lists all the illustrations and figures used in the work, including the caption and the corresponding page numbers (preset in this template). The list of figures is the first list with an independent, otherwise Roman section numbering. This list only needs to be appended when necessary. In individual cases, it may be helpful to consult the work supervisor.

\subsection*{Optional: II List of Tables}
The list of tables lists all the tables used in the work, including their labels and the corresponding page numbers (preset in this template). This list only needs to be appended when necessary. In individual cases, it may be helpful to consult the work supervisor.

\subsection*{Optional: III List of Formulas}
The list of formulas lists all formulas used in the work, including the label and the corresponding page number. It is only necessary if there are at least 3 or more formulas used in the work. This list only needs to be appended when necessary. In individual cases, it may be helpful to consult the work supervisor.

\subsection*{Optional: IV List of Abbreviations}
The list of abbreviation lists all abbreviations used in the work, including their definition. This list only needs to be appended when necessary. In individual cases, it may be helpful to consult the work supervisor.

\subsection*{Optional: V List of Symbols}
The list of symbols lists all mathematical symbols used in the work, including their definition. This list only needs to be appended when necessary. In individual cases, it may be helpful to consult the work supervisor.

\subsection*{1 Introduction}
Within the introduction, the objective is formulated, placed in an superordinate context and distinguished from other topics. The most important terms of the topic must be placed within the context to be considered in the introduction while an thorough formulation is particularly important. In addition, information on the methodology can be provided. The structure of the work should be presented. 

Thus the first section contains the following aspects:

\begin{itemize}
	\item Description of the problem to be solved with this work
	\item Objective of the work and the delivered contribution to science
	\item Structure of the work
\end{itemize}

In the case of extensive scientific work, such as bachelor's and master's theses, the introduction should consist of one to three pages. In addition, it is useful to insert a figure at the end of the section which shows the structure, the argumentative sequence or important core statements of the work. Experience shows that the introduction should only be formulated in detail at the end of the work in order to avoid repeated changes to the main text.

\subsection*{2 Related Work}
Based on the introductory explanations, this section focuses on the contextualisation of the problem or objective described and the synthesis of the necessary fundamental principles. Necessary fundamental principles contain knowledge on topics that cannot be assumed to be engineering knowledge in the field of construction sciences or computer science.

In this context it is important to explain elementary terms, definitions and/or theories on the one hand and to present the state of research in the respective subject area on the other hand.
Accordingly, a summary is given of the scientific basis on which the scientific work to be written is based.

\subsection*{3 Methodology}
This section focuses on the methodology chosen to achieve the research objectives or to answer the research questions. It is made clear how the objective of the work formulated in the introduction is to be achieved or implemented using scientific methods.

\subsection*{4 Additional Main Section(s)}
In general, this main section or sections represents your results. The concrete structure of the scientific work depends largely on the task to be solved or the research questions to be answered, so that no detailed recommendations for this section can be given here, but exemplary sections could be requirements analysis, conception, development, evaluation for a development work or test setup, test execution, evaluation for an experimental work. For the specific structuring of the main section(s) a consultation with the supervisor of the work is helpful in any case.

\subsection*{5 Discussion}
The discussion serves to interpret or critically discuss the results achieved on the basis of the problem or question formulated in the introduction as well as in the context of current and relevant literature on the topic. In this context, unexpected results or contradictions to assumptions made or to the existing scientific literature will be specifically worked out.

\subsection*{6 Conclusion}
The conclusion completes the scientific work as a final statement. The objective, the methodology applied and the conclusions drawn are summarised once again and assessed in terms of the extent to which the introductory objective could be achieved. It is advisable to consider possible limitations of your work and to reflect critically on your approach. In addition, you should formulate in an outlook how further scientific work could tie in with its results and which priorities you see for further research work in the topic area of your work.

\subsection*{Bibliography}
In the bibliography, all sources used in the context of scientific work are specified in detail.

\subsection*{Appendix}
In principle, only those contents are included in the appendix which are not essential for the understanding of the content, but which may have an important relation to the own elaborations or an additional information character. These include, for example, questionnaires specially developed for the purpose of a study.


  \section{Stand der Technik}
Um dieses Konfigurationsproblem zu automatisieren oder zumindest zu erleichtern, gibt es bereits verschiedene Lösungen. Momentan sind drei Frontend-Frameworks besonders beliebt (React, Angular und Vue) \cite{stateofjs:fe_framework_usage} und für jedes dieser Frameworks existiert jeweils ein Programm, was (unter anderem) zur initialen Erstellung von Projekten mit dem jeweiligen Framework empfohlen wird.

\subsection{React}
In der Dokumentation für React wird empfohlen, neue Projekte über das \gls{CLI} \gls{CRA} zu erstellen. Dieses kann per \gls{npm} installiert werden und ist dann in der Lage, den Inhalt eines angegebenen Templates (oder des Standardtemplates) in ein angegebenes Verzeichnis zu kopieren. Daraufhin werden die benötigten Abhängigkeiten per \gls{npm} oder mittels eines anderen installierten Paketmanagers (wie z.B. Yarn) installiert.

Dritten ist es möglich, eigene Templates für \gls{CRA} zu erzeugen, die dann wie Erstanbietertemplates zur Erzeugung des neuen Projekts genutzt werden können. Wenn man also mittels \gls{CRA} ein Projekt erzeugen möchte, das bereits mit gewissen Bibliotheken oder Werkzeugen ausgestattet ist, muss dafür dafür ein Template erstellt worden sein. Die Wahrscheinlichkeit, dass das jedoch passiert ist, sinkt mit der Spezifität der Wünsche.

\subsection{Angular}
Das Angular-Team stellt das sog. Angular-CLI zur Verfügung, das Entwickelnden beim Programmieren verschiedene wiederkehrende und repetitive Aufgaben abnehmen soll (wie z.B. die Erstellung und Einbindung neuer Komponenten). Neben diesen Aufgaben wird das Tool auch zur Erstellung neuer Angular-Projekte genutzt.

Der Befehl ng new <project-name> führt Nutzende in einen Dialog, bei dem einige Fragen zur gewünschten Konfiguration des Projektes gestellt werden. Es können vier Features konfiguriert werden und daraufhin wird das Projekt eingerichtet.

Im Gegensatz zu CRA sind hier die Möglichkeiten, die Nutzenden geboten werden, sehr eingeschränkt, da es nicht wie bei den Templates für Dritte die Möglichkeit gibt, neue Libraries mitsamt entsprechender Konfiguration in den Erstellungsprozess einzubinden. Dafür können sich Nutzende hier (im Rahmen der beschränkten angebotenen Optionen) eine beliebige Konfiguration aussuchen und sind nicht darauf angewiesen, dass jemand vor ihnen schon denselben Wunsch hatte. Außerdem wird so auch Anfänger:innen die Entdeckung und der Einstieg in neue Libraries erleichtert.


\subsection{Vue}
Von den drei Frameworks bietet Vue in Bezug auf die Projekterstellung das umfangreichste CLI. Hier trifft man zunächst eine Vorauswahl von zehn Features, die man haben oder nicht haben möchte. Daraufhin werden zu den ausgewählten Features detailliertere Fragen gestellt. Insgesamt stehen einem durch dieses Tool über 20 verschiedene Libraries ohne weiteren Konfigurationsaufwand zur Verfügung.


\subsection{Vergleich der Möglichkeiten zwischen Frameworks}
Alle drei Tools bieten Entwickelnden die Möglichkeit, eigene Erweiterungen zu erarbeiten und zu veröffentlichen. Im Falle von React muss dies in Form eines Templates geschehen. Da nur ein Template zur Erstellung eines Projektes genutzt werden kann, sind derartige Erweiterungen hier also exklusiv.

Bei Angular und React ist es jedoch möglich, auch Erweiterungen zu entwickeln, die zusätzlich zu anderen Optionen und Erweiterungen nutzbar sind. Für das Angular \gls{CLI} kann man sogenannte Schematics entwickeln, die das Hinzufügen und Einbinden einer Bibliothek vollautomatisch übernehmen. Diese Schematics müssen aber leider nach der Installation ausgeführt werden und müssen insbesondere von Nutzenden entdeckt werden. Hierfür gibt es keine eigene Plattform o.ä. und die Auswahl der Schematics, die schon bei der Projekterstellung auswählbar sind, beschränkt sich auf Angular-interne Features (z.B. die Einfügung eines Routers). Selbst Libraries, die auch vom Angular-Team betreut werden (z.B. Angular Material) müssen später per Schematic nachgerüstet werden.

Im Rahmen des Vue \gls{CLI}'s ist es immerhin möglich, auch Libraries von Dritten direkt bei der Projekterstellung einzubinden. Allerdings ist auch hier die anfängliche Auswahl nicht erweiterbar. Dafür können, wie auch schon bei Angular, anschließend automatisch über Drittanbieterplugins neue Libraries heruntergeladen, importiert und demonstriert werden.

Tabelle 

  \begin{table}
	  \centering
	  \caption{Automatische und initiale Installierbarkeit verschiedener Features in Angular und Vue Projekten}
	  \begin{tabular}{|l|l|l|l|l|}
    \hline
         & \multicolumn{2}{c|}{Angular} & \multicolumn{2}{c|}{Vue}  \\ \hline
        Feature & Automatisch & Ititial & Automatisch & Initial \\
        & installierbar & auswählbar & installierbar & auswählbar \\ \hline
        TypeScript & \multicolumn{2}{c|}{wird erzwungen} & \checkmark & \checkmark \\ \hline
        Router & \checkmark & \checkmark & \checkmark & \checkmark \\ \hline
        PWA-Support & \checkmark & \texttimes & \checkmark & \checkmark \\ \hline
        Linter & \checkmark & \texttimes & \checkmark & \checkmark \\ \hline
        Formatierer & \checkmark & \texttimes & \checkmark & \checkmark \\ \hline
        CSS Reset & \texttimes & \texttimes & \texttimes & \texttimes \\ \hline
        CSS Präprozessor & \checkmark & \checkmark & \checkmark & \checkmark \\ \hline
        Design Framework & \checkmark & \texttimes & \checkmark & \texttimes \\ \hline
        State Management & \multicolumn{2}{c|}{wird erzwungen} & \checkmark & \checkmark \\ \hline
        Unit Testing Framework & \checkmark & \texttimes & \checkmark & \checkmark \\ \hline
        E2E Testing Framework & \checkmark & \texttimes & \checkmark & \checkmark \\ \hline
    \end{tabular}
	  \label{tab:automatically_installable_libs_per_framework}
  \end{table}

Außerdem bieten alle drei der genannten Tools keine Unterstützung für andere Frameworks. Daher ist beispielsweise der große Umfang der Features des Vue CLI‘s leider nicht in einem Angular-Projekt nutzbar.






  \section{Konzeptionierung}
\label{komzeptionierung}
Angesichts dessen, dass das geplante \gls{CLI} verschiedenste Bilbiotheken und Werkzeuge unterstützen können soll und dabei auch zukunftssicher bleiben soll, ist es unabdingbar, die Entwicklung im Vorhinein zu planen. Hierfür muss zunächst untersucht werden, welche Gemeinsamkeiten und welche Unterschiede bei verschiedenen Installationsprozessen vorliegen. Daraufhin kann dann ein System erarbeitet werden, was die Gemeinsamkeiten fördert und gleichzeitig Raum für die bekannten, aber auch möglichst für erwartbare unbekannte Unterschiede lässt.

\subsection{Auswahl der zu vergleichenden Installationsprozesse}
Um zu bestimmen, welche Schritte bei der Einrichtung von neuen Projekten durchführbar sein müssen, wurden zunächst verschiedenste Projekte erstellt, zu denen dann möglichst verschiedene Abhängigkeiten und Werkzeuge hinzugefügt wurden.

Aufgrund der bereits erläuterten Beliebtheit von Frontend-Frameworks wurde für jedes der drei bekanntesten Framework ein Projekt erstellt, in dem dann die entsprechenden Installationen durchgeführt wurden. Für diese initiale Erstellung wurde jeweils das bereits vorgestellte \gls{CLI} verwendet.

Bei der Auswahl dessen, was zu diesen Projekten hinzugefügt wurde, lag der Fokus zum einen auf Bekanntheit und Verbreitung und zum anderen auf Vielseitigkeit. Da sich ähnliche Tools oft auch in ihrem Installationsprozess ähneln (beispielsweise waren die Schritte zur Installation von ESLint fast identisch zu den Schritten zur Installation von TSLint\footnote{\url{https://palantir.github.io/tslint/}}), ist es für diese Vorbereitung also nicht nötig, eine Abhängigkeit hinzuzufügen, wenn bereits eine andere ausprobiert wurde, die einen Ähnlichen Funktionsumfang hat.

Aus diesen Gründen wurden folgende Werkzeuge installiert:

\begin{itemize}
\item \textbf{ESLint} ist ein Werkzeug zur statischen Codeanalyse für JavaScript. Es gibt Warnungen aus, wenn bestimmte Ausdrücke verwendet wurden, die häufig zu Bugs führen (z.B. die Verwendung von einem doppelten Gleichheitszeichen anstelle eines dreifachen Gleichheitszeichens, d.h. die typunsichere Gleichheitsüberprüfung anstelle Gleichheitsprüfung mit Prüfung auf Typgleichheit).

ESLint verfügt auch über die Möglichkeit, Formatierungspräferenzen anzugeben. Diese werden in die Analyse mit einbezogen und darüber können beispielsweise Warnungen ausgegeben werden, wenn Zeilen falsch eingerückt sind. Einige Warnungen, darunter die meisten Warnungen zur Formatierung, können von ESLint auf Wunsch automatisch behoben werden. Standardmäßig ist ESLint jedoch nur passiv, also idempotent.

ESLint wird stellvertretend für alle Tools zur statischen Codeanalyse überprüft. Insbesondere entfällt deshalb die Überprüfung von TSLint; zumal das Projekt veraltet ist und die Entwicklung zugunsten von ESLint aufgegeben wurde \cite{tslint_deprecation} \cite{tslint_repo}. Da es aber noch automatisch mit Angular zusammen installiert wird, hätte man seine Überprüfung dennoch in Betracht ziehen können.

\item \textbf{Prettier} ist ein Codeformatierer, der unter freiwilliger Angabe von Konfigurationseigenschaften Code einheitlich formatiert (d.h. Einrückungen werden korrigiert, Leerzeichen vor oder hinter Klammern eingefügt oder entfernt etc.). Gegenüber ESLint hat Prettier den Vorteil, nicht nur JavaScript-basierte Dateien formatieren zu können, sondern es kann auch mit anderen Dateiformaten wie Markdown oder CSS umgehen.

Darüber hinaus kann Prettier auch ohne Konfiguration (d.h. ohne Festlegung von Präferenzen) aufgerufen werden um Code zu formatieren. ESLint hingegen muss ausgiebig konfiguriert werden, damit der Code formatiert wird. Zudem hat Prettier mehr Einstellmöglichkeiten als ESLint (zumindest in Bezug auf Formatierung).

\item \textbf{SCSS / Sass}\footnote{\url{https://sass-lang.com/}} sind zwei Dialekte eines CSS-Präprozessors. Dieser Präprozessor erweitert die CSS-Syntax um eine weitere Art von Variablen, Verschachtelung, sogenannten "Mixins" und weitere Features. Zur Kompilierzeit wird daraus normales CSS erzeugt. Dank der zusätzlichen Funktionen kann kompakteres, wiederholungsfreieres und oft auch leserlicheres CSS geschrieben werden.

Zugunsten von SCSS / Sass wurde auf die weitere Überprüfung anderer Präprozessoren wie Less\footnote{\url{https://lesscss.org/}} verzichtet.

\item \textbf{Jest} ist eine Bibliothek zum Schreiben und Ausführen automatisierter Tests. Diese werden in Node.js ausgeführt und verfügen daher a priori nicht über Browserspezifische APIs (wie z.B. Zugriff auf Canvas-Elemente). Viele dieser Funktionalitäten sind aber ggf. über weitere Bibliotheken Nachrüstbar. In der Regel wird Jest für Unittests verwendet.

Im Vergleich zu anderen Tools vereint Jest mehrere Funktionalitäten, die sonst über mehrere Abhängigkeiten aufgeteilt sind. Wo sonst der Ausführer der Tests von der Bibliothek zur Aufstellung und Überprüfung von Annahmen (Engl.: \glqq Assertion Library\grqq) und der Mockingbibliothek getrennt ist und zur Erzeugung eines Berichtes über die Testabdeckung eine weitere Bibliothek notwendig ist, bietet Jest alle diese Features gleichzeitig an.

Seit 2019 ist Jest das meist verbreitetste Werkzeug zum Schreiben von JavaScript-Tests \cite{stateofjs} und unterstützt dabei alle drei Frameworks. Außerdem ist seine Installation vergleichsweise sehr simpel, da hier nicht mehrere Tools zusammen installiert und miteinander eingerichtet werden müssen. Da diese höhere Komplexität also vermeidbar ist, werden andere Library-Kombinationen für Unittests nicht näher betrachtet.

\item \textbf{Cypress} ist ein weiteres Werkzeug zum Schreiben automatisierter Tests, allerdings mit dem wichtigen Unterschied zu Jest, dass die Tests im Browser ausgeführt werden. Aktuell ist Cypress in der Lage, Chromium-basierte Browser und Firefox zu kontrollieren, um das Laden und die Interaktion mit einer Seite nicht nur per Node.js zu simulieren, sondern entsprechende Ereignisse in einem Browser so auszulösen, dass sie nicht (in relevantem Ausmaß) von tatsächlichen Userinteraktionen unterschieden werden können. Außerdem stehen in diesen Tests die Browser-APIs in vollem Umfang zur Verfügung.

Seit 2020 wird Cypress öfter als vergleichbare Projekte wie Puppeteer oder WebdriverIO verwendet \cite{stateofjs}, weshalb diese zugunsten von Cypress nicht weiter berücksichtigt werden.

\item \textbf{Husky und lint-staged} ist eine Kombination von Tools, die die Ausführung anderer Tools als Reaktion auf bestimmte Ereignisse ermöglichen.

Einige der zuvor aufgeführten Tools bieten Leistungen an, die dabei helfen können, sicherzustellen, dass in ein \gls{VCS} eingepflegter Code immer gewissen Standards entspricht. Die automatischen Tests können beispielsweise garantieren, dass kein zuvor existierendes (und von Tests abgedecktes) Feature kaputt gemacht worden ist, und Prettier und ESLint könne garantieren, dass der Code einem gewissen Stil entspricht. Andere Werkzeuge wie CSS-Präprozessoren hingegen bieten in dieser Hinsicht keine weiteren Vorteile.

Um die Qualität des Codes auf einem möglichst hohen Niveau zu halten, kann es also sinnvoll sein, diese entsprechenden Tools automatisch vor der Einpflegung des Codes in ein \gls{VCS} ausführen zu lassen. Genau diese Möglichkeit bietet Husky, sofern das \gls{VCS} Git verwendet wird. Diese Annahme lässt sich durchaus treffen, da Git mit einem Abstand von ??? das am weitesten verbreitete \gls{VCS} ist \missingQuote.

Mithilfe von Husky lassen sich mit wenig Konfigurationsaufwand Skripte festlegen, die vor dem Speichern von Code in Git ausgeführt werden sollen. Scheitert eines dieser Skripte, so wird der Speichervorgang abgebrochen. Lint-staged erleichtert hierbei das Überprüfen der geänderten Dateien, sodass nicht bei jeder Änderung der gesamte Code geprüft werden muss.

Aufgrund dessen, dass diese beiden Tools empfohlen werden, um Prettier automatisch vor entsprechenden Vorgängen laufen zu lassen, und da ihre Verwendung sich leicht auf zusätzliche Tools erweitern lässt, werden Husky und lint-staged vergleichbaren Alternativen vorgezogen.
\end{itemize}

Außerdem wurden die folgenden Bibliotheken installiert:

\begin{itemize}
\item \textbf{Router} - Jedes der drei Frameworks verfügt über eine Unterbibliothek um verschiedene Routen einzurichten. So kann man ohne ein Neuladen der Seite die URL wechseln, diese Änderung auch im Browserverlauf widerspiegeln lassen und auch schon beim Aufruf einer Unter-URL direkt die entsprechende Komponente anzeigen lassen. Schon die Tatsache, dass die Installation eines Routers in Angular- und Vueprojekten schon bei der initialen Einrichtung ausgelöst werden kann, zeigt, dass Router sehr häufig verwendet werden. Für ein jeweiliges Framework gibt es außerdem in der Regel genau einen geläufigen Router, sodass sich hier keine besondere Bevorzugung stattgefunden hat. Stattdessen wurden alle drei Router gleichermaßen untersucht.

%Für React muss das \gls{npm}-Paket "react-router-dom" (und, falls TypeScript gewünscht ist, die zugehörigen Typdefinitionen) installiert werden. Daraufhin muss man die Hauptkomponente mit einer bestimmten Komponente umgeben, die vorher zu importieren ist. In neu zu erstellenden Dateien kann man dann Routen definieren, die dann wieder in der Hauptkomponente importiert und eingebunden werden müssen.
%
%Die Installation eines Routers für Angular und Vue ist erheblich einfacher: hier muss lediglich bei der initialen Installation angegeben werden, dass der jeweilige Router installiert und initialisert werden soll.

\item \textbf{Redux} - Wie bereits erläutert, verfügt React nicht über eine besonders empfohlene Methode zum zentralen Statemanagement. Diese Lücke kann jedoch von Redux gefüllt werden. Mithilfe von Redux kann ein zentraler sogenannter "Store" erzeugt werden. Dieser kann modifiziert werden, indem man an anderer Stelle eine sogenannte "Action" erzeugt, die an den Store weitergeleitet wird. Daraufhin werden sogenannte "Reducer" aufgerufen, die den aktuellen Wert des Stores und die aktuelle Action entgegennehmen und einen neuen, entsprechend der Action aktualisierten Store zurückgeben.

Diese Reducer sind stets pure Funktionen\footnote{D.h. sie sind seiteneffektfrei und ergeben bei gleicher Eingabe immer die gleiche Ausgabe.} und sind daher (da sie keinen Initialisierungsprozess voraussetzen und nichts anderes beeinflussen) sehr leicht testbar. Außerdem ermöglicht Redux in Kombination mit einer Browsererweiterung mehrere interessante Features, darunter ein Feature namens "Zeitreise", mit dem man den Store zu einem beliebigen vorherigen Stand zurücksetzen kann.

Redux lässt sich als eine Implementierung des Kommandopatterns \missingQuote\ betrachten. Hierbei stellen die Actions die Kommandos dar und der Store ist der Ausführer, der auch die Kommandohistorie verwaltet. Entsprechend stellt Redux auch automatisch eine Aufwandslose Undo-Redo-Funktionalität zur Verfügung.

Da Redux die meistverwendete Bibliothek für Statemanagement ist \cite{stateofjs}, wird ihre Installation stellvertretend für Alternativen wie MobX betrachtet. Ausnahme hiervon bildet jedoch Vuex, da es eine sehr hohe Ähnlichkeit zu Redux aufweist, sich mit besonders wenig Aufwand in einem Vue-Projekt installieren lässt und es für Vue-Projekte anstelle von Redux empfohlen wird \cite{vuejs_docs:redux_vs_vuex}.

\item \textbf{Komponentenlibraries} - Gerade für kleine Projekte, in denen keine eigenen Designer angestellt sind, kann es sich sehr lohnen, eine Kombonentenlibrary einzubinden. Da diese jeweils für ein Framework Komponenten bereitstellen muss, gibt es in der Regel für verschiedene Design Systeme jeweils eine Bibliothek pro Framework.

Stellvertretend für andere Komponentenlibraries wurden drei Material-Design-Libraries untersucht: Material UI für React, Angular Material für Angular und Vuetify für Vue.

\item \textbf{Paper.js} - Um abschließend noch eine Bibliothek zu untersuchen, die an sich keinerlei Bindung an Frameworks hat aber trotzdem in die Applikation eingebunden werden muss, wurde willkürlich als Beispiel paper.js ausgewählt. Diese Library erleichtert den Umgang mit dem HTML Canvas.
\end{itemize}

\subsection{Vergleich der Installationsprozesse}
Beim manuellen Installieren der verschiedenen Tools und Libraries sind einige Gemeinsamkeiten und vor allem in den Details viele Unterschiede aufgefallen. Auf einer allgemeineren, konzeptuellen Ebene lässt sich aber vor allem (wie bereits geschehen) zwischen Werkzeugen und Bibliotheken unterscheiden.

Allgemeinhin müssen für beides zunächst \gls{npm}-Pakete installiert werden. Daraufhin benötigen aber Werkzeuge meist die Änderung von Konfigurationsdateien, während zur Einbindung von Bibliotheken eher der produktive Code modifiziert werden muss.

\subsubsection{Installationen von Werkzeugen}
Zur Installation von Prettier ist beispielsweise lediglich noch die Erstellung einer Konfigurationsdatei für Prettier notwendig. Theoretisch kann auch auf diese Verzichtet werden, aber da ein Ziel des Projektes ist, dass Nutzende alles Wichtige für die meiste Arbeit eingerichtet bekommen, ist es sinnvoll, die Datei bereits zu erstellen und mit sinnvollen Standardwerten zu füllen. Darüber hinaus ist es ratsam, zur leichteren Ausführung von Prettier (wie bei den meisten anderen Werkzeugen auch) ein \gls{npm}-Script zu erzeugen. Hierfür muss die package.json-Datei modifiziert werden können.

Ähnlich ist bei der Installation von ESLint vorzugehen. In der Regel bedarf es lediglich der Installation des Paketes, der Einrichtung eines entsprechenden Scripts und der Erzeugung der Konfigurationsdatei. Leider wird der Installationsprozess bei der Benutzung von Angular dadurch verkompliziert, dass hier standardmäíg zunächst das veraltete TSLint eingebunden ist. Allerdings gibt es bereits ein \gls{npm}-Paket\footnote{\url{https://github.com/angular-eslint/angular-eslint}}, was den Wechsel auf ESLint automatisch durchführen kann.

Mit überdurchschnittlich wenig Aufwand lässt sich Sass / SCSS installieren. Sowohl Angular als auch Vue verfügen hierzu über einen Parameter bei der initialen Installation. Bei React muss man zwar das node-sass Paket installieren, aber weitere Einrichtung muss nicht vorgenommen werden.

Überdurchschnittlich viel Aufwand verursachten dahingegen Jest und Cypress. In Kombination mit React wird Jest zwar automatisch installiert und im Vue-\gls{CLI} gibt es hierfür eine entsprechende Option. Allerdings setzt Angular bisher auf Karma als Testausführer und Mocha als Assertion Library, weshalb hier nicht nur eine Installation von Jest sondern auch eine Anpassung bestehender Scripte und alter Tests notwendig ist. Wie auch schon bei ESLint gibt es ein Paket, was den Umstieg bis auf die Aktualisierung der Tests selber automatisch durchführen kann. Da im Kontext dieses Projekts bei der Ausführung noch keine speziellen Tests entstanden sein können, kann diese Aufgabe hier durch ein schlichtes Austauschen von festgelegten, vorgefertigten Dateien erledigt werden.

Die Installation von Cypress ist vergleichsweise zwar simpel und benötigt lediglich das Installieren neuer \gls{npm}-Pakete, das erstellen eines Scripts in der package.json-Datei und das Erzeugen neuer Dateien. Dieser Prozess kann im Kontext von Vue sogar von der initialen Projekterstellung abgenommen werden und auch in Angular-Applikationen kann er mithilfe eines \gls{npm}-Pakets automatisiert werden. Allerdings ergibt sich in genau der Kombination (d.h. in Angular-Projekten) eine Komplikation, wenn auch Jest installiert wurde. Dieses Problem lässt sich durch Modifikation der jest.config.js-Datei beheben \cite{angular_jest_cypress_issue}.

Da das Schreiben von Test sowohl für Jest als auch für Cypress sowohl in JavaScript als auch in TypeScript erfolgen kann, sollten die automatisch eingerichteten Tests in derjenigen der beiden Sprachen sein, die auch für den Produktivcode verwendet wird. Es ist also bei der Installation notwendig zu wissen, ob TypeScript verwendet wird oder nicht.

Eine letzte Besonderheit ergibt sich bei der Installation von Husky und lint-staged. Hier sind zwar ähnliche Schritte wie bei anderen Tools notwendig, aber zusätzlich wird hier eine Liste von Befehlen bzw. Scripten benötigt, die vor bestimmten Git-Prozessen ausgeführt werden sollen.

Zusammenfassend ist für die Installation von Tools also eine Kenntnis über die Verwendung von TypeScript nötig. Es muss auch bekannt sein, ob unter den anderen ausgewählten Werkzeugen solche dabei sind, die vor Git-Prozessen ausgeführt werden sollen. Darüber hinaus ist die Installation von \gls{npm}-Paketen sowie die Modifikation einiger Dateien notwendig. Diese Dateien sind entweder im \gls{JSON}-Format oder sind JavaScript- / TypeScript-Dateien. Außerdem muss es möglich sein, neue Dateien mit vorbestimmtem Inhalt zu erzeugen, der sich an den JavaScript-Dialekt anpassen können muss.

\subsubsection{Installationen von Bibliotheken}
Auch bei der Installation von Bibliotheken gibt es viele Gemeinsamkeiten, aber hier sind deutlich größere Unterschiede zwischen den Frameworks zu vermerken. Dies ist vorallem dadurch zu begründen, dass die Benutzung der Bibliotheken innerhalb bestimmter Komponenten geschieht und daher entweder bestehende Komponenten modifiziert oder neu erstellte Komponenten eingebunden werden müssen. Aufgrund dieser Bindung an Komponenten muss das entsprechende Vorgehen an jedes Framework einzeln angepasst werden.

Ein gutes Beispiel hierfür ist die Installation von paper.js. Da es sich hier nur um eine spezielle Library handelt, ist sie die einzige, die in keinem Framework vorinstallierbar ist, und eignet sich daher besonders gut für diesen Vergleich. Zur sinnvollen Benutzung von paper.js ist es notwendig, einen Canvas auf der Webseite anzuzeigen. Sobald dieser Canvas existiert, kann paper.js initialisiert werden und etwas auf dem Canvas zeichnen (zu Demonstrationszwecken genügt ein einfaches Rechteck).

Hierfür lässt sich in allen drei Frameworks eine Komponente schreiben, die dann in die Hauptkomponente eingebettet werden muss. Die dafür erforderlichen Dateiveränderungen sind jedoch sehr verschieden, was am Beispiel von Vue in Listing \ref{code:vue:use_component} dargestellt wird.

%\begin{lstlisting}[caption={Die zu ändernden Zeilen um eine neue Komponente in React einzubinden.}, captionpos=b, language=diff, label={code:react:use_component}]
%  // src/App.tsx
%  import React from 'react'
%+ import { PaperJsExample } from './PaperJsExample'
%  // ...
%
%  function App() {
%    return (
%      <div className="App">
%        <header className="App-header">
%          {/* ... */}
%        
%+         <PaperJsExample />
%        
%          {/* ... */}
%        </header>
%      </div>
%    )
%  }
%
%export default App
%\end{lstlisting}
%
%\begin{lstlisting}[caption={Die notwendigen Änderungen um dasselbe in Angular zu erreichen.}, captionpos=b, language=diff, label={code:angular:use_component}]
%  <!-- app.component.html -->
%  
%  <!-- ... -->
%  
%  <div class="content" role="main">
%  
%    <!-- ... -->
%  
%+   <app-paper-js-example></app-paper-js-example>
%  
%    <!-- ... -->
%  </div>
%  
%  <!-- ... -->
%\end{lstlisting}

\begin{lstlisting}[caption={In Vue notwendige Änderungen, um eine Komponente hinzuzufügen}, captionpos=b, language=diff, label={code:vue:use_component}]
  // HelloWorld.vue
  <template>
    <div class="hello">
      <!-- ... -->
      
+     <PaperJsExample />
      
      <h3>Installed CLI Plugins</h3>
      <!-- ... -->
    </div>
  </template>
  
  <script lang="ts">
  import { defineComponent } from "vue";
+ import PaperJsExample from "@/components/PaperJsExample.vue";
  
  export default defineComponent({
    name: "HelloWorld",
    props: {
      msg: String,
    },
+   components: {
+     PaperJsExample,
+   },
  });
  </script>
  
  <!-- ... -->
\end{lstlisting}

Wie man hier sieht, muss die zu verwendende Komponente zum einen dort eingefügt werden, wo sie anschließend im \gls{DOM} erscheinen soll. Zum anderen muss sie im Script-Block importiert werden. Danach folgt eine Besonderheit von Vue: es muss deklariert werden, dass die Komponente verwendet werden wird (vgl. Zeilen 22 - 24 in Listing \ref{code:vue:use_component}). Dieses Attribut des Konfigurationsobjektes muss bei der ersten Komponente, die eingefügt werden soll, neu erzeugt werden, während es bei allen weiteren neuen Komponenten nur ergänzt werden muss.

Dieses eine Codebeispiel zeigt bereits, dass es schon für simple Installationen von Bibliotheken notwendig ist, framework-spezifischen Code modifizieren zu können. Hierfür wäre es hilfreich, ein tiefes Verständnis des zu modifizierenden Codes zu haben. Da aber die Struktur des vorher existierenden Codes bekannt ist (unter der Annahme, dass sich die initialen Setups nicht auf relevante Art und Weise verändern werden) und da die vorzunehmenden Änderungen sich innerhalb eines Frameworks immer ähneln, kann diese geringe Menge von Änderungen bzw. Änderungsarten auch ohne ein solches tiefes Verständnis umgesetzt werden. Genauer wird hierauf in Kapitel \ref{implementierung} eingegangen.

Die Komponentenlibraries lassen sich mehrheitlich ohne bereits bekannte Probleme einbinden; häufig sogar automatisch bei der initialen Projekterzeugung mittels des Framework-spezifischen \gls{CLI}'s.

Neue Probleme weisen jedoch die Einbindung von Redux und Routern auf. Diese Libraries bauen beide darauf auf, dass gewisse Informationen global verfügbar sind (in Redux der Store und in Routern die aktuelle Route).

In React ist das umsetzbar, indem die oberste Komponente von sogenannten "Providern" umgeben wird, die einen Wert und einen zugehörigen Schlüssel entgegennehmen und dann jeder ihnen unterliegenden Komponente den Wert unter dem jeweiligen Schlüssel zur Verfügung stellen. Das Umgeben der obersten Komponente ist also notwendig, damit alle Komponenten der gesamten Applikation Zugriff auf diese Werte haben. Falls dies nicht gewünscht ist, kann der zugehörige Provider verschoben werden.

In Angular ist es für eine sinnvolle Verwendung von Redux sinnvoll, einen Angular-Service einzurichten. Dies ermöglicht die einfache Erstellung eines Singleton \missingQuote Store-Objektes und erlaubt es gleichzeitig, mit Angular-typischen Mitteln wie RxJS\footnote{\url{https://rxjs.dev/}}-Methoden auf den Store zuzugreifen. Im Grunde genommen handelt es sich hierbei aber erneut nur um das erstellen einer Datei mit vorbestimmtem Inhalt.

Im Zusammenhang mit Vue fallen keine weiteren Probleme auf, da hier für Redux Vuex substituiert wird und sowohl Vuex als auch der Vue Router bereits automatisch installierbar sind.

\subsubsection{Zusammenfassung der notwendigen Installationsschritte}
Über die Werkzeuge und Bibliotheken hinweg zeigt sich, dass sich vieles über einfache Befehle installieren lässt, ohne, dass ein weiterer Aufwand betrieben werden muss. Für andere Installationen ist es ausnahmslos notwendig, gegebene Pakete von \gls{npm} installieren zu können. Es muss möglich sein, neue Dateien an bestimmten Orten zu erzeugen und diese mit Inhalten zu füllen, die sich nach der getroffenen Auswahl von Werkzeugen und Bibliotheken richten können.

Außerdem muss die Modifikation bestimmter Dateien ermöglicht werden. Zum einen muss es möglich sein, verschiedene \gls{JSON}-Dateien zu verändern. Zum anderen muss frameworkspezifischer Code ergänzt und modifiziert werden können. Außerdem muss die Reihenfolge der Installation der Tools bzw. Werkzeuge festlegbar sein, da beispielsweise Husky und lint-staged als letzte Werkzeuge installiert werden sollten.

\subsection{Wahl des generellen Vorgehens}
\subsection{Planung der Erweiterungen}
\subsection{Abwägung über Sonderstellung für TypeScript und Frameworks}

  \section{Implementierung}
\subsection{Abhängigkeiten und Exklusitiväten}
Im Rahmen der Abhängigkeiten und Exklusivitäten von Erweiterungen zueinander gibt es mehrere Probleme, die gelöst werden müssen. Zum einen kann es schnell passieren, eine unmögliche Kombination von Bedingungen zu stellen. Außerdem muss nach der Auswahl einer Kombination von Erweiterungen geprüft werden, ob diese Kombination zulässig ist.

\subsubsection{Überprüfung der Umsetzbarkeit}


Ein triviales Beispiel hierfür wäre ein Paar von Abhängigkeiten $A$ und $B$, wobei $A$ und $B$ exklusiv zueinander sind und gleichzeitig $A$ von $B$ abhängt. Es können in diesem Beispiel niemals beide Anforderungen gleichzeitig erfüllt werden.

Es lassen sich aber auch kompliziertere Beispiele erzeugen. In Fig A wird eine Konstellation dargestellt, bei der $A$ von $B$ und $B$ von $C$ abhängt, aber $A$ zu $C$ exklusiv ist. Diese Kombination von Anforderungen lässt sich ebensowenig wie das erste Beispiel gleichzeitig erfüllen, ist aber komplizierter zu erfüllen. Die Ursache des Problems ist, dass Abhängigkeiten transitiv sind. Sind also die Erweiterungen $A$ von $B$ und $B$ von $C$ abhängig, so ist auch indirekt $A$ von $C$ abhängig.

Die bisherigen Beispiele zeigen Probleme auf, bei denen eine Erweiterung zugleich (indirekt) abhängig und exklusiv zu einer anderen Erweiterung ist. Es gibt jedoch auch eine weitere Kategorie von Problemen: sind zwei (indirekte) Abhängigkeiten einer Erweiterung zueinander exklusiv, so sind ebenfalls nicht alle Anforderungen gleichzeitig umsetzbar.

Aus diesen fachlichen Überlegungen heraus ergibt sich eine erste Lösung. Für jede Erweiterung $E$ sind zunächst alle direkten sowie indirekten Abhängigkeiten zu bestimmen. Für alle diese Abhängigkeiten ist dann zu prüfen, dass zum einen die Abhängigkeit $A$ nicht zu $E$ exklusiv ist, aber auch dass $A$ zu keiner weiteren (indirekten) Abhängigkeit $A'$ von $E$ exklusiv ist.

Bei näherer Betrachtung dieser Lösung lässt sich jedoch einiges an Ineffizienz feststellen. Zum einen werden für jede Erweiterung erneut die transitiven Abhängigkeiten berechnet. Aufgrund eben dieser Transitivität sind jedoch die Berechnungen für alle Erweitungen redundant, die Abhängigkeit einer anderen Erweiterung sind. Im schlimmsten Fall (nämlich wenn eine von $n$ Erweiterung alle anderen als (indirekte) Abhängigkeiten hat) ist also eine Berechnung nötig und $n - 1$ Berechnungen werden unnötigerweise durchgefüht. Aus ähnlichem Grund ist auch die Überprüfung der Exklusitivät zweier (indirekter) Abhängigkeiten häufig überflüssig.

Die Lösung dieser Probleme ergibt sich aus einer theoretischeren Betrachtung des Problems. Wie bereits aus der Verwendung des Begriffs der Transitivität hervorgeht, lassen sich Abhängigkeit und Exklusitivät als mathematische Relationen über der Menge aller Erweiterungen auffassen. Hierbei ist besonders hervorzuheben, dass die Exklusivität symmetrisch ist (ist $A$ zu $B$ exklusiv, so ist auch $B$ zu $A$ exklusiv) während Abhängigkeit nicht symmetrisch ist (im Gegenteil: bei der anfänglichen Analyse von möglichen Bibliotheken, Frameworks etc. ergab sich, dass Abhängigkeit nie symmetrisch zu sein scheint). Allerdings ist Exklusivität a priori nicht transitiv (auch, wenn $A$ zu $B$ und $B$ zu $C$ exklusiv ist, können $A$ und $B$ zusammen verwendet werden), während Abhängigkeit sehr wohl transitiv ist (wie bereits erläutert).

Vor diesem Hintergrund lässt sich erkennen, dass die Bestimmung der (indirekten) Abhängigkeiten der Bestimmung der Transitiven Hülle gleichkommt. Diese kann mittels des Floyd-Warshall-Algorithmus (in der Warshall-Variante) berechnet werden \cite{warshal1_algorithm}. Hierfür muss zunächst ein gerichteter Graph erzeugt werden, in den alle bekannten Abhängigkeiten als Kante eingefügt werden. Von diesem Graphen wird dann die transitive Hülle bestimmt, in der dann zwei Knoten $A$ und $B$ genau dann durch eine (von $A$ nach $B$ gerichtete) Kante verbunden sind, wenn die zugehörige Erweiterung $A$ von $B$ abhängt.

Auch die Relation der Exklusivität lässt sich in einen Graphen überführen. Auch in diesem Graphen gibt es pro Erweiterung einen Knoten und jede Exklusivität wird als Kante dargestellt. Aufgrund der Symmetrie der Exklusivität kann dieser Graph aber ungerichtet sein. Das Problem der Überprüfung der Restriktionen reduziert sich nun darauf, sicherzustellen, dass es für jede zwischen zwei Knoten in maximal einem der beiden Graphen eine Kante gibt, wobei die Richtung keine Rolle spielt (denn wenn die Knoten exklusiv zueinander sind, darf keiner von dem anderen abhängig sein).

Anders formuliert, darf es im Graphen der Exklusivitäten keine Kante $\{A, B\}$ geben, für die in der transitiven Hülle der Abhängigkeiten die Kante $(A, B)$ oder die Kante $(B, A)$ existiert. Aufgrund dessen, dass die transitive Hülle gerichtet ist, gibt es darin doppelt so viele Kanten, wie in dem Graphen der Exklusivitäten, weshalb diese Richtung einfacher zu prüfen ist, als die Umkehrung.

Der Algorithmus von Floyd-Warshall sorgt dafür, dass diese Überprüfung eine asymptotische Laufzeit von $\mathcal{O}(n^3)$ hat. Da die oben beschriebenen Probleme in beliebiger Tiefe von Abhängigkeiten auftreten können, ist jedoch die Bestimmung der transitiven Hülle nicht vermeidbar. Allerdings ist damit zu rechnen, dass die Anzahl aller Erweiterungen stets kleiner als $100$ sein wird (andernfalls würde die Benutzbarkeit des Programms möglicherweise stark eingeschränkt). Daher ist diese Laufzeit in diesem Fall als unbedenklich einzustufen.
  \section{Evaluierung}
\label{eval}
Die Feststellung des Erfolgs der Implementierung erfolgt in zwei Schritten. Zunächst wird in Bezug auf die bei der Konzeptionierung erarbeiteten Features betrachtet, ob alle gewünschten Funktionalitäten implementiert wurden. Daraufhin ist anhand bestimmter Kriterien zu überprüfen, ob die implementierten Features funktionieren.

\subsection{Evaluierung der umgesetzten Features}
In Kapitel \ref{konz:all_features} ist eine vollständige Liste aller gewünschter Features erfasst worden. Diese umfasst vor allem Features, die für Nutzende von Interesse sind, erwähnt aber auch Aspekte der Erweiterbarkeit und Vorbereitung auf zukünftige Entwicklung. Im folgenden wird untersucht, ob all diese Ziele erreicht werden konnten.

Das erste erwähnte Feature ist die interaktive und informative Auswahl von Bibliotheken und Werkzeugen. Es konnte ein Fragebogen umgesetzt werden, der genau eine solche Auswahl ermöglicht. Vor und nach dieser Selektion sind weitere Fragen möglich, wobei bereits über \gls{CLI}-Argumente beantwortete Fragen nicht erneut gestellt werden, sodass die Konfiguration standardmäßig komplett interaktiv ist, aber auf Wunsch auch ohne jegliche Interaktion stattfinden kann.

Aufgrund der Typdefinition von Erweiterungen ist es unmöglich, eine Erweiterung zu \gls{GWA} hinzuzufügen, die nicht über grundlegende Metadaten verfügt, worunter insbesondere der Name der Erweiterung, ihre Beschreibung sowie ein Link zu weiteren Informationen zählen. Somit sind die Ziele der Interaktivität und der Informativität als erfüllt zu betrachten.

Bei der Umsetzung der Fragen war außerdem darauf zu achten, dass keine unzulässige Konfiguration auswählbar ist. Dieses Feature konnte umgesetzt werden, indem nach jeder Frage die gegebene Antwort überprüft und ggf. zurückgewiesen wird. Nutzende haben im Fall einer unzulässigen Antwort dann weiterhin die Möglichkeit, ihre Antwort zu bearbeiten.

Es wäre schöner gewesen, Nutzenden bereits während der Beantwortung jeder Frage Rückmeldung zu der Gültigkeit der aktuell geleisteten Antwort zu geben. Da die für den Dialog verwendete Bibliothek jedoch kein solches Feature anbietet, wird stattdessen bei jeder Frage und ggf. neben jeder Antwortmöglichkeit angegeben, welche Einschränkungen gelten. Nutzende können basierend auf diesen Informationen versuchen, unzulässige Antworten zu vermeiden, werden aber zusätzlich beim Versuch der Einreichung jeder Antwort daran gehindert, fehlerhafte Antworten zu geben.

Diese Validierung findet auch statt, falls bestimmte Antworten schon im Rahmen der \gls{CLI}-Argumente gegeben werden. Dabei fehlschlagende Validierungen führen jedoch zu einem fehlerhaften Beenden von \gls{GWA}, um eine automatische Benutzung zu ermöglichen. Diese Entscheidung hat sich bereits bei der Erstellung automatischer Tests als hilfreich erwiesen. Auch das gewünschte Feature der Antwortvalidierung wurde also erfüllt.

Ebenfalls umgesetzt werden konnten die Auswählbarkeit des Paketmanagers sowie die Installation über Framework-spezifische \gls{CLI}s wo sie möglich ist. Damit kann, wie bereits bei der Konzeptionierung erläutert, die zukünftige Pflegbarkeit der erzeugten Projekte erleichtert werden, da die für das jeweilige Framework empfohlenen Werkzeuge genutzt werden können.

Die Umsetzung von Erweiterungen konnte nur etwas eingeschränkter geschehen. Zwar wurden mehrere verschiedene Kategorien von Erweiterungen umgesetzt, darunter Werkzeuge wie Frameworks und CSS-Präprozessoren, aber auch Bibliotheken wie Redux. Allerdings ist ist darauf verzichtet worden, Erweiterungen für Werkzeuge und Bibliotheken zum automatischen Testen zu schreiben.

Diese Erweiterungen hätten sich von den implementierten Erweiterungen deutlich unterschieden, da sie einen großen Einfluss auf andere Erweiterungen gehabt hätten. Bei ihrer Installation hätten nämlich nicht nur die entsprechenden Werkzeuge / Bibliotheken installiert und eingebunden werden müssen, sondern es hätten sämtliche bisher existierenden Tests auf die ausgewählte Erweiterung angepasst werden müssen. Darüber hinaus hätten auch, sofern sinnvoll, von allen anderen Erweiterungen Tests basierend auf den ausgewählten Werkzeugen und Bibliotheken erzeugt werden müssen. Beispielsweise müsste demnach die Redux-Erweiterung für die erzeugten Komponenten ebenfalls Tests generieren.

Aufgrund der hohen Komplexität dieses Features und dem begrenzten, im Rahmen dieser Arbeit leistbaren Arbeitsumfangs wurde vorerst darauf verzichtet, Erweiterungen für automatische Tests umzusetzen. Diese Erweiterungen sind jedoch für die weitere Entwicklung von \gls{GWA} geplant und werden daher im Kapitel \ref{further_research} erneut thematisiert werden.

Obwohl dies nicht explizit als Feature gewünscht wurde, wurde außerdem bei der Entwicklung sehr darauf geachtet, dass der Code von Dritten erweiterbar und veränderbar ist. Dies ist vor allem durch automatische Tests gewährleistet worden.

\subsection{Sicherstellung der Funktionalität}
Die Überprüfung der Korrektheit einiger Features, insbesondere der Installation von Erweiterungen, ist nur mit hohem Aufwand ausführlich möglich. Für jede Erweiterung müsste betrachtet werden, welche Installationsschritte sie durchführen soll. Häufige Schritte wie die Installation eines \gls{npm}-Pakets sind leicht zu überprüfen, aber bei außergewöhnlicheren Schritten wie der Ergänzung einer Komponente im Falle der Redux-Erweiterung müsste die erzeugte Applikation lokal gestartet werden und es müsste von Hand verifiziert werden, dass die Komponente eingefügt worden ist und über ihre erwartete Funktionalität verfügt.

Außerdem werden einige Features durch die Auswahl bzw. die Nicht-Auswahl anderer Erweiterungen aktiviert bzw. deaktiviert. Es müssen zur vollständigen Überprüfung solcher Erweiterungen also mehrere Projekte eingerichtet und überprüft werden.

Aufgrund dieses hohen Umfangs wären manuelle Tests mit einem großen zeitlichen Aufwand verbunden. Außerdem wäre die Durchführung der Tests an vielen Stellen gleich und daher anfällig für Flüchtigkeitsfehler.

Gleichzeitig ist die Automatisierung dieser Tests aufgrund ihrer Vielfältigkeit nur schwer möglich. Für viele Erweiterungen müssten spezielle Tests angefertigt werden und die Überprüfung bestimmter Features (wie die Funktionalität der durch die Redux-Erweiterung eingebundene Komponente) ließe sich am besten über Ende-zu-Ende-Tests mithilfe entsprechender Werkzeuge realisieren. Genau solche Tests sollen künftig über Erweiterungen installierbar sein, aber dann müsste man diese Tests zu diesen Evaluierungszwecken auch dann in Projekte einbauen können, wenn sie eigentlich gar nicht Teil der zu testenden Konfiguration sein sollen.

Während der manuellen Tests, die im Rahmen der Entwicklung durchgeführt wurden, hat sich jedoch gezeigt, dass sich die meisten solcher detaillierter Fehler gut durch Unit-Tests abdecken ließen. Die danach übrig gebliebenen Fehler waren ausnahmslos derart schwerwiegend, dass entweder der Installationsprozess oder das Bauen der resultierenden Applikation fehlgeschlagen hat.

Auf diese Beobachtung hin wurde eine vereinfachte Prüfung von Installationen erarbeitet, die sich leicht automatisieren lässt. Eine zu prüfende Konfiguration wird an \gls{GWA} weiter gegeben. Im Anschluss an eine erfolgreiche Installation wird das Projekt gebaut und sofern diese beiden Prozesse fehlerfrei laufen gilt das Projekt als erfolgreich eingerichtet.

Mithilfe von Docker können beliebige Konfigurationen innerhalb eines Containers durchgeführt werden, also in einem Kontext mit isoliertem Datei- und Betriebssystem aber im Gegensatz zu einer \gls{VM} ohne isolierten Kernel. Ein TypeScript-Skript orchestriert die Erzeugung und Löschung der Container sowie die Ausführung der Installations- und Baubefehle innerhalb der Container. Im Anschluss an die automatisch durchgeführten Befehle wird jeder Container gespeichert, damit die getroffene Installation für weitere manuelle Tests zur Verfügung steht.

In diesem Skript kann festgelegt werden, welche Konfigurationen zu überprüfen sind. Da während der Entwicklung viele Probleme durch die Kombination bestimmter Erweiterungen entstanden sind, werden bei diesen Tests alle Erweiterungen in möglichst umfangreichen Projekten getestet. Unter Beachtung von Abhängigkeiten und Exklusivitäten werden also maximal viele Erweiterungen zusammen installiert.

Zum Zeitpunkt des Schreibens dieser Arbeit liegen vor allem\footnote{Außerdem ist Less nicht zusammen mit React verwendbar. Diese Kombination wird im späteren Verlauf einfach ausgelassen.} zwei Arten von Exklusivitäten vor: die zwischen Frontend-Frameworks und die zwischen CSS-Präprozessoren. Um trotz dieser Exklusivitäten alle möglichen Konstellationen überprüfen zu können, wird jede (zulässige) Kombination eines Frameworks und eines CSS-Präprozessors überprüft.

Eine weitere große Fehlerquelle war die Installation von Erweiterungen mit oder ohne TypeScript, da abhängig von TypeScript weitere \gls{npm}-Pakete zu installieren waren, besonderer Code generiert werden musste oder ähnliches. Eine Ausnahme bilden hierbei jedoch die CSS-Präprozessoren. Daher wird mit ihrer Ausnahme jede Erweiterung sowohl mit, als auch ohne TypeScript überprüft.

Aus diesen Gründen werden insgesamt acht Konstellationen von Erweiterungen überprüft: Angular wird in Kombination mit jedem der vier CSS-Präprozessoren (hier wird die Abwesenheit eines CSS-Präprozessors auch als Präprozessor gezählt) und allen sonstigen Erweiterungen installiert. Da Angular von TypeScript abhängt, werden diese Konfigurationen nie ohne TypeScript überprüft. Außerdem wird React mit jedem CSS-Präprozessor außer Less und allen übrigen Erweiterungen überprüft. Dazu kommt lediglich eine Konfiguration ohne TypeScript, da die Anwesenheit von TypeScript keinen Einfluss auf CSS-Präprozessoren hat.

Durch das Durchführen dieser Tests sind mehrere Fehler aufgefallen, die vor allem in der Kombination von CSS-Präprozessoren mit React aufgetreten sind und die vollständig behoben werden konnten. Bei der weiteren manuelle Überprüfung der dabei erstellten Projekte sind keine weiteren Fehler aufgefallen. Abschließend sind also keine Fehler bei bisher implementierten Features bekannt.

\subsection{Alternative Lösungswege}
Für einige der Ansätze, die in dieser Arbeit beschrieben und umgesetzt wurden, sind nach der Umsetzung alternative Lösungswege aufgefallen. Aufgrund der zeitlichen Einschränkungen konnten diese nicht mehr umgesetzt werden, aber dennoch werden sie im folgenden erläutert und mit den bisherigen Ansätzen verglichen.

Der erste solche Fall ist der Umgang mit nicht-installierten Paketmanagern. In dem hier erläuterten Ansatz sind diese bei der Wahl des zu verwendenden Paketmanagers nicht auswählbar. Um Nutzende aber zumindest über die Existenz anderer Paketmanager zu informieren, werden diese dennoch angezeigt.

Da \gls{GWA} ohne Node.js nicht ohne großen Mehraufwand ausführbar ist, ist davon auszugehen, dass \gls{npm} immer installiert ist. Yarn, der andere unterstützte Paketmanager, ist über \gls{npm} installierbar. Somit wäre es betriebssystemunabhängig möglich, beide Paketmanager anzubieten und Yarn ggf. nachzuinstallieren. Dieser Ansatz hätte mit einem ähnlichen Aufwand wie die Validierung bzw. die Deaktivierung der entsprechenden Option durchgeführt werden können. Aufgrund der späten Idee war dies aber im Rahmen dieser Arbeit nicht mehr möglich.

Eine weitere alternative Herangehensweise war bei der Verwendung von inquirer Möglich. Inquirer muss aufgrund der asynchron gestellten Fragen asynchron aufgerufen bzw. verwendet werden. Dies ist einerseits über die Verwendung von in JavaScript integrierte Promises möglich. Andererseits kann dieselbe Asynchronität mit RxJS\footnote{\url{https://github.com/reactivex/rxjs}} erreicht werden.

RxJS ist eine Bibliothek, die im Kern das Observer-Pattern implementiert. Sie stellt Observables und Subjects zur Verfügung, zwischen denen der in diesem Zusammenhang signifikanteste Unterschied ist, dass an ein Subject neue Ereignisse weitergereicht werden können, was bei einem Observable nicht möglich ist. Das Subject implementiert jedoch das Observable.

Zu Beginn der Implementierung von \gls{GWA} wurde aufgrund eines Missverständnisses inquirer zusammen mit RxJS verwendet. Dafür wurde zunächst ein Subject erzeugt, an das alle zu stellenden Fragen weitergeleitet wurden. Von Inquirer wurde dann ein Observable entgegengenommen, dem die Antworten der Fragen entnommen werden konnten.

Dieser Ansatz führte zu deutlichen Komplikationen. Neben dem in Kapitel \ref{inquirer_type_issue} beschriebenen Problem mussten manuell alle Antworten ihren jeweiligen Fragen zugeordnet werden. Zudem mussten nach der Stellung von Fragen durch eine Erweiterung die zugehörigen Antworten weitergeleitet werden, ohne weitere Antworten weiterzuleiten. Dies wäre mit dem anderen Ansatz nicht nötig gewesen, da hier der Stellung jeder Frage automatisch die zugehörige Antwort zugeordnet werden konnte.

Nach der Lösung all dieser Probleme ist das Missverständnis aufgefallen. Aufgrund der beschriebenen Komplikationen und dieses Missverständnis ausreichend früh aufgefallen ist, konnte noch zu dem anderen Ansatz gewechselt werden. Dennoch hätte durch Ausprobieren beider Ansätze das Missverständnis früher aufgelöst und somit viel Komplikation vermieden werden können.
  \section{Weitere Forschung}
\label{further_research}

\end{document}
